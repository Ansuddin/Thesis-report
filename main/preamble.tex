\documentclass[a4paper,10pt,oneside,openany]{memoir}

% Aesthetics
\chapterstyle{hangnum}
\renewcommand\contentsname{Table of Contents}

% Set language
\usepackage[british]{babel}

% bibliography
\usepackage[numbers]{natbib}

% Packages
\usepackage{xcolor}
\usepackage[T1]{fontenc}
\usepackage[utf8]{inputenc} % unicode support
\usepackage{tocbibind}
\usepackage{graphicx}
\graphicspath{ {figures/} }
\usepackage{pdfpages} % to import the front page
\usepackage{import}
\usepackage[justification=centering]{caption}% or e.g. [format=hang]
\usepackage{verbatim}
\usepackage{amsmath}

\usepackage{geometry} %% Alter margins on pages

\usepackage{adjustbox}

\usepackage{caption}
\usepackage{subcaption}
\usepackage{tabularx}
\usepackage{color}
\usepackage[explicit]{titlesec}
\usepackage{lmodern}
\usepackage{fixltx2e}

\usepackage{adjustbox}
\usepackage{array}

\newcolumntype{R}[2]{%
	>{\adjustbox{angle=#1,lap=\width-(#2)}\bgroup}%
	l%
	<{\egroup}%
}
\newcommand*\rot{\multicolumn{1}{R{45}{1em}}}

\newlength\mylen
\setlength\mylen{4em}

\setsecnumdepth{subsubsection}

\titleformat{\chapter}
{\HUGE\sffamily}{\llap{\makebox[\mylen][l]{{\thechapter}}\hfill}}{0em}{#1}

\titleformat{\section}
{\Huge\sffamily}{\llap{\makebox[\mylen][l]{{\thesection}}\hfill}}{0em}{#1}

\titleformat{\subsection}
{\LARGE\sffamily}{\llap{\makebox[\mylen][l]{{\thesubsection}}\hfill}}{0em}{#1}

\titleformat{\subsubsection}
{\Large\sffamily}{\llap{\makebox[\mylen][l]{{\thesubsubsection}}\hfill}}{0em}{#1}

\titleformat{\paragraph}
{\Large\sffamily}{\theparagraph}{1em}{#1}

\titlespacing*{\paragraph}
{0pt}{3.25ex plus 1ex minus .2ex}{1.5ex plus .2ex}


% Set margin 
\setlrmarginsandblock{4.25cm}{4.25cm}{*}
\setulmarginsandblock{4.25cm}{*}{1}
\checkandfixthelayout 

% To list code blocks in latex
\usepackage{lstautogobble}
\usepackage{listings}
\lstset{
	language=C,                % choose the language of the code
%	numbers=left,                   % where to put the line-numbers
	stepnumber=1,                   % the step between two line-numbers.        
%	numbersep=5pt,                  % how far the line-numbers are from the code
	backgroundcolor=\color{white},  % choose the background color. You must add \usepackage{color}
	showspaces=false,               % show spaces adding particular underscores
	showstringspaces=false,         % underline spaces within strings
	showtabs=false,                 % show tabs within strings adding particular underscores
	tabsize=2,                      % sets default tabsize to 2 spaces
	captionpos=b,                   % sets the caption-position to bottom
	breaklines=true,                % sets automatic line breaking
	breakatwhitespace=true,         % sets if automatic breaks should only happen at whitespace
%	title=\lstname,                 % show the filename of files included with \lstinputlisting;
	autogobble=true
}

\lstdefinestyle{ttt}{
    basicstyle=\ttfamily,
  %columns=fullflexible,
    mathescape=true,
    literate={->}{$\rightarrow $}{1}
  }

\lstdefinestyle{csv}{
	xleftmargin=-2.5cm,
	xrightmargin=-2.5cm,
	belowskip=0pt,
	basicstyle=\scriptsize,
	stringstyle=\scriptsize,
	fontadjust=true}
% xml listings
\lstdefinestyle{XML}{ 
	columns=fullflexible, 
	basicstyle=\rmfamily, 
	commentstyle=\ttfamily\itshape\color{green!50!black}, 
	morestring=[s]{"}{"}, 
	morecomment=[s]{?}{?}, 
	morecomment=[s]{!--}{--}, 
	morecomment=[s]{!DOCTYPE}{]}, 
	moredelim=[s][\color{black}]{>}{<}, 
	moredelim=[s][\bfseries\color{black}]{\ }{=}, 
	stringstyle=\color{blue}, 
	identifierstyle=\bfseries\color{violet} 
}


%%%%%%%%%%%%%%%
% json listings starts
%%%%%%%%%%%%%%%%
\newcommand\JSONnumbervaluestyle{\color{blue}}
\newcommand\JSONstringvaluestyle{\color{red}}

% switch used as state variable
\newif\ifcolonfoundonthisline

\makeatletter

\lstdefinestyle{json}
{
	showstringspaces    = false,
	keywords            = {false,true},
	alsoletter          = 0123456789.,
	morestring          = [s]{"}{"},
	stringstyle         = \ifcolonfoundonthisline\JSONstringvaluestyle\fi,
	MoreSelectCharTable =%
	\lst@DefSaveDef{`:}\colon@json{\processColon@json},
	basicstyle          = \ttfamily,
	keywordstyle        = \ttfamily\bfseries,
}

% flip the switch if a colon is found in Pmode
\newcommand\processColon@json{%
	\colon@json%
	\ifnum\lst@mode=\lst@Pmode%
	\global\colonfoundonthislinetrue%
	\fi
}

\lst@AddToHook{Output}{%
	\ifcolonfoundonthisline%
	\ifnum\lst@mode=\lst@Pmode%
	\def\lst@thestyle{\JSONnumbervaluestyle}%
	\fi
	\fi
	%override by keyword style if a keyword is detected!
	\lsthk@DetectKeywords% 
}

% reset the switch at the end of line
\lst@AddToHook{EOL}%
{\global\colonfoundonthislinefalse}

\makeatother

%%%%%%%%%%%%%%%
% json listings ends
%%%%%%%%%%%%%%%%

\usepackage{float}
\usepackage[hidelinks]{hyperref}


% glossaries
\usepackage{glossaries}
\newacronym{mad}{GCHQ}{The Government Communication Headquarters} % list of glossaries in a separate file

\newcounter{quote}

\newcommand\citea[1]{%
	(\citeauthor{#1},~\citeyear{#1})}

\newcommand\citeap[2]{%
	    \ifx&#1&%
	    % #1 is empty
	    (\citeauthor{#2},~\citeyear{#2})
	    \else
	    % #1 is nonempty
	    (\citeauthor{#2},~\citeyear{#2}:#1)%%
	    \fi
}
	
% enquote some text
\newcommand{\q}[1]{``#1''}
\newcommand{\qe}[1]{``\emph{#1}''}

% Commands
% Horizontal lines used for the Front Page Title
\newcommand{\HRule}{\rule{\linewidth}{0.5mm}}