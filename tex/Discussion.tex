The previous section applied Paragon as a programming language and demonstrated how policies can be defined and enforced. Hence achieving stronger security guarantees. This section considers what Information-Flow Control offers to the prototype and discusses Paragon from a programmers perspective.

\section{The Policy system}
Matrix primary use case is as a secure messaging protocol however it has a wide range of use cases. If Matrix is used as a secure communication channel like in the use case for IoT described in section \ref{endtoend}, then the end-to-end encryption is not enough to achieve confidentiality and integrity throughout the system. Information-Flow Control is a mechanism that aid in achieves stronger security guarantee.


The prototype displays major improvements in terms of confidentiality and integrity in the system. The prototype is secure by construction and gives a guarantee that the defined policies about the information is enforced. Consider the Java code where a secretary handles a patient journal: 

\begin{lstlisting}

public class PatientJournal{

private String publicNote;
private String[] publicSessions; // Unaccessible to secretary
private String[] privateSessions; // Unaccessible to secretary

}

public class Secretary {

public void receive(PatientJoural journal){
// Perform task on journal
}

\end{lstlisting}

The Java compiler would be helpless in detecting if the secret parts of the journal are unintentionally accessed or modified. In the prototype such unintentional access or modification of information would be detected immediately by the Paragon compiler. This is a strong guarantee that Information-Flow Control tools offer.

The prototype also demonstrate improvement to an actual problem in the current journal systems. The section xx describes how the only mechanism is auditing and logging trail and clearly a mechanism for enforcing policies would be of great benefit. 

The prototype is by no means a full-fledged journal system and has some obvious limits. The prototype lacks support for concurrency and assumes that only a single user can use it at a time. The prototype does not handle the issue with concurrent writes described \ref{concurrentwrites} which is also impractical. The purpose of the prototype was to demonstrate improvements of the security guarantees provided by matrix to ensure end-to-end security by enforcing policies which the prototype has demonstrated.


\section{Matrix}
The interface allows Matrix to be used with an Information-Flow Control tool.
% Hvor fyldestgørende/brugbart er interfacet?
%Husk i Discussion, at komme tilbage hertil. Eventuelt opsummér Matrix security model, med det slut, at end-to-end encryption ikke er tilstrækkeligt; at dit API gør, at man nu kan kode end-to-end security ovenpå Matrix.

%Du havde nogen gode referencer, der motivérede dit systemdesign og valget af Matrix (i forhold til Danmarks ønsker/vision/plan). Jeg synes ikke jeg har set henvisninger til det endnu i læsningen.



\section{Paragon}

% Policy first approach or code first approach

%Reflekter over Paragon som værktøj. Hvad er godt og hvad er mindre godt 
% Mindre teknisk forbedring genering af .pi interface filer.
% Hvor godt vil locks virke med concurrency? 
% Would be a problem in a concurrent system then the Referred lock would be shared by multiple users at the same time. Another design would be to have a list of global users or what? 



% The relation between actor and entity in the system might be blurry 
% The same policies can be defined several ways. Which way is best? The prototype type takes a more dynamic labeling approach.
% The integrity policy could be build upon with actors as well but 

% The policies interferes with eachother if the same actor is used in multiple policies.
% Hence using an entity as actor 
% Perhaps better way of specifying what the policy context for some object is?
% Declassifying methods? 
% More expressiveness when putting "must open" locks signature

% Should be carefully considered when locks should be opended and when declassification and as well as endorsement should occur. The 


% 


\section{Summary}
