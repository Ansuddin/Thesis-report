This chapter describes the implementation of the prototype in the case study. The prototype is a system for sending and retrieving patient journals among different hospitals. The system relies on Matrix as the secure communication channel and storage.

\section{Journal system}

% Describe the requirements for the journal system.
A journal system serves an important purpose by providing patient journals to different hospitals and clinics. If a patient arrives at the ER and the doctor cannot access the patient's journal then the treatment of the patient gets problematic. A doctor might miss out on important details about the patient or even worse give medication that might give the patient an allergic reaction. The availability of a patient journal is a necessity however the number of medical employees that have access to such a journal has raised privacy concerns. Consider the scenario where a patient gets referred to a physiotherapist with muscle pain. When the therapist opens the journal the full medical history will be present; if the patient had received psychiatric treatment those session would be readable too. Furthermore a patient journal is accessible by a large number of unrelated medical employees with the only prevention mechanism being logging and audit trails.


The lack of secure information is evident and the prototype demonstrates how Information-Flow control can be leveraged to enforce security policies concerning the information. The prototype is a simple journal system where multiple hospitals can receive and send journals with secure information flow at the endpoints. The use cases that the prototype implements are presented next.

\subsection{Requirements}

Overall requirements:
- Hospitals needs shared access to patient journal. 

- Each hospital has a homeserver running.

- A room represents a patient journal with the participants in the room being the respective hospitals that share the journal. 

- The global Patient journal state is stored in matrix.

- Employees in a hospital shares the same matrix login.

- Patient journals are append only.


- A hospital has two principals: Doctor and Secretary.

- A doctor can see full patient journal.

- The doctor must have the patient in care to access the high information in a journal.

- A secretary can only see some parts of the journal.

- The must be 

- If the hospital is an ER any doctor should access it 


The medical staff can only retrieve patient journals from the hospital.





\subsection{Matrix}




Configurations for room:
- Rooms are precreated and each hospital are aware of each room. 
- Hospitals in a Room are preconfigured and every client knows from the beginning which hospitals are in the room / or can join the room.
- The room is invite-only
- Default user level is 100. Anyone can kick, ban and invite users. 



\subsection{System design}

A central authority (the government) would be managing the room which all participants in the room trust. The room can be considerably large since many different types of clinics and hospitals needs access to a patient journal. This puts a lot of responisbility on securing the endpoints. If a clinic or hospital is present in a room it means that the patient has been referred to that clinic. We assume that all emergency hospitals have access to the room.
% Describe how the design with matrix would be


\subsubsection{Limitations}

A design problem is the concurrent writes to a journal from multiple hospital. When writing to a journal the latest version of the journal is first retrieved and then the writes are appended to the journal. However multiple hospitals might have retrieved the latest journal and different doctors might have committed changes to the journal and send it to Matrix hence one of the writes would be lost since only the latest journal is retrieved.

This is a common issue in distributed systems but is not handled in this solution.

% Solution merging data like version control systems like git.



\section{Paragon implementation}

\subsection{The Matrix interface}

\subsection{Policies and locks}

\subsubsection{Lattice}

\subsection{Declassification}

\subsection{Exception handling}


\subsection{Limitations}

The Matix 
Technical requirements:
-Msg type should be "m.text" when sending patient journal (JSON format). The convertion of string to JSON and vice versa should happen in the client.

\section{Summary}